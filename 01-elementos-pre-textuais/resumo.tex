% -----------------------------------------------------------------------------
%   Arquivo: ./01-elementos-pre-textuais/resumoPt.tex
% -----------------------------------------------------------------------------



\begin{resumo}
Para a solução de problemas de otimização complexos, o desenvolvimento de meta-heurísticas e suas combinações são, em geral, de  difícil abordagem, pois é necessário um grande conhecimento do problema. Uma alternativa ao desenvolvimento de novas meta-heurísticas, ou a hibridização manual delas, é utilizar os mecanismos de colaboração e comunicação próprios da modelagem de sistemas multi-agentes (MMAS). A arquitetura D-Optimas é um MMAS baseado no modelo de atores, onde cada agente encapsula uma meta-heurística diferente e, dotado de um mecanismo de aprendizagem colabora com os demais agentes para encontrar a melhor solução para um problema de otimização. Os agentes interagem no espaço de busca que é divido em regiões, que possuem um comportamento independente, podendo receber novas soluções, se particionar ou se fundir. Entretanto, como a execução da arquitetura D-Optimas está limitada à um número pequeno de nós em um \textit{cluster}, estendê-la, permitindo a execução em um \textit{cluster} com um número indeterminado de nós, é essencial para a resolução de problemas em larga escala e extração de dados confiáveis necessários na análise de desempenho da arquitetura. Portanto, o objetivo deste trabalho é consolidar a arquitetura D-Optimas do ponto de vista de um sistema distribuído, tolerante a falhas, com balanceamento de carga e transparência de localidade, tornando-a resiliente e escalável horizontalmente. Os resultados obtidos por esta pesquisa até o momento mostram que a arquitetura escala de maneira eficiente em um \textit{cluster} com até seis nós sem perda de desempenho. Espera-se, com o prosseguimento deste trabalho, estudar a hibridização dinâmica da arquitetura com uma grande variedade de agentes.

\textbf{Palavras-chave}: Sistemas multi-agente; Otimização; Sistemas distribuídos.
 

\end{resumo}


\begin{resumo}[Abstract]
Solving complex optimization problems using meta-heuristics and their combinations are, in general, a hard approach, since a great knowledge of the problem often is necessary. An alternative to the development of new algorithms, or their manual hybridization, is to use the collaboration and communication mechanisms inherent in the modeling of multi-agent systems (MMAS). The D-Optimas architecture is an MMAS based on the actor model, where each agent encapsulates a different meta-heuristic, and with a learning mechanism they are allowed to collaborate in finding the best solution to an optimization problem. The agents interact in the search space that is divided into regions, which have an independent behavior, being able to receive new solutions, spliting into three new regions or merge with another one. However, the execution of the D-Optimas is limited to a small number of nodes in a cluster. Extending it, allowing its execution in a cluster with an unfixed number of nodes is essential for solving large-scale problems and extracting reliable data to analyze the architecture's performance. Therefore, this work aims to consolidate the D-Optimas architecture from the point of view of a distributed system, fault tolerant, with load balancing and location transparency, making it resilient and horizontally scalable. The results obtained by this research so far shows that the architecture scales efficiently in a cluster with up to six nodes without losing performance. It is expected, with the continuation of this work, to study the dynamic hybridization of architecture with a wide variety of agents.


\textbf{Keywords}: Multi-agent systems; Optimization; Distributed systems.
\end{resumo}